\documentclass[twocolumn]{article}

\usepackage{aaai}
\usepackage{times}
\usepackage{helvet}
\usepackage{courier}

\newcommand{\fix}[1]{{\bf #1}}

\setcounter{secnumdepth}{2}

\title{Playing to Program:  An Intelligent Programming Tutor for RUR-PLE}
\author{Marie desJardins and an Ensemble of Students \\
University of Maryland, Baltimore County \\ 
1000 Hilltop Circle \\
Baltimore MD  21250 \\
mariedj@cs.umbc.edu
}


\begin{document}
\maketitle

\begin{abstract}
Intelligent tutoring systems (ITSs) have proven their effectiveness in
contributing to student learning.  ITSs are automated programs
that provide students with a one-on-one tutor, allowing them to work
at their own pace, so they can spend more time on their weaker areas
of the subject matter.  RUR--Python Learning Environment (RUR-PLE), a
virtual environment to help 
students learn to program in Python, provides an interface for
students to write their own Python code and then be presented with a
visualization of that same code [CITE].  
The RUR-PLE system provides a sequence of learning lessons for
students to explore.  We have extended RUR-PLE to provide an
intelligent tutoting system interface that consists of three
components:  (1) a
student model that tracks student understanding, (2) a diagnosis module
that provides tailored feedback to students, and (3) a problem selection
module that guides the student's learning process.  In this paper, we
describe the basis RUR-PLE system and our extensions, and present the
results of a user study in which we evaluated the effectiveness of our
three ITS modules.
\end{abstract}

\section{Introduction and Motivation}
\label{sec:intro}

\fix{ADD introduction and motivation.}

\section{Related Work}
\label{sec:related-work}

\fix{ADD: A discussion of general work on ITS, programming tutors,
CS1/iteration education research, etc.}

\section{Background: RUR-PLE}
\label{sec:background}

\fix{ADD general overview of RUR-PLE and the functionality
it provides.}

\section{Infrastructure}
\label{sec:student-model}

\fix{ADD description of concept map, how we built/tested it 
(i.e., justification for these concepts and why they're
connected as they are, and how we instantiate and track it for a
specific user (presumably using some kind of Bayesian updating).  
Idea:  validate/finalize it by some testing process on
a group of students (i.e., is it in fact the case that students in
general need to understand concept X before they can apply concept Y)
-- use a problem suite (where each problem includes known concepts)
to measure these dependencies.} 


\section{Pre-test}
\label{sec:diagnosis}

\fix{ADD:  Discussion of diagnosis module:  our state-based
approach to comparing student solutions to one or more model
solutions per problem.  Where do the model solutions come from?
What does the comparison consist of?  How is the resulting
``diff'' used to generate suggestions (and update the student model)?}


\section{Problem Selection}
\label{sec:prob-selection}

\fix{ADD: How is the student model used to generate and select
  problems for the student to work on?  What is the motivation for our
  approach, and how does it work?}

\section{Experimental Design}
\label{sec:exper-design}

\fix{ADD:  Methodology:  design of the user study (set up in such a

\section{Resultes}
\label{sec: resultes}

\fix{ADD: Resultes}


\section{Conclusions and Future Work}
\label{conclusions}

\fix{ADD:  What have we contributed?  What are the take-away lessons?
  What would we work on next?}

\section*{References}


\end{document}
